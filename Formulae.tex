\documentclass{article}

\usepackage[margin=0.75in]{geometry}

\title{\textbf{Mathematical Formulae}}
\author{Nakul Singh}

\usepackage{amsmath}
\usepackage{amsfonts}
\usepackage{amssymb}

\usepackage{fancyhdr}


\begin{document}

\begin{titlepage}
\clearpage\maketitle
\thispagestyle{empty}
\begin{center}
\textit{Formatted and Type-setted using}\\
\medskip
{\Large\LaTeX}
\end{center}
\end{titlepage}
\newpage

\pagestyle{fancy}
\fancyhead[L]{Mathematical Formulae}
\fancyhead[R]{\LaTeX}
\fancyfoot[C]{Page \thepage}
\twocolumn

\section{Trigonometry}

%\subsection{Identities:}
%$
%sin^{2}\theta+cos^{2}=1\\
%sec^{2}\theta-tan^{2}\theta=1\\
%cosec^{2}\theta-cot^{2}\theta=1
%$

\subsection{Addition/Difference Formulae:}
$
\indent\indent sin(A+B)=sinA\,cosB+cosA\,sinB\\
\indent\indent \smallskip sin(A-B)=sinA\,cosB-cosA\,sinB\\
\indent\indent cos(A+B)=cosA\,cosB-sinA\,sinB\\
\indent\indent \medskip cos(A-B)=cosA\,cosB+sinA\,sinB\\
\displaystyle
\indent\indent \smallskip tan(A+B)=\frac{tanA+tanB}{1-tanA\,tanB}\\
\indent\indent \medskip tan(A-B)=\frac{tanA-tanB}{1+tanA\,tanB}\\
\indent\indent \smallskip cot(A+B)=\frac{cotA\,cotB-1}{cotB+cotA}\\
\indent\indent \medskip cot(A-B)=\frac{cotA\,cotB+1}{cotB-cotA}\\
\indent\indent \smallskip sinC+sinD=2\,sin\left(\frac{C+D}{2}\right)cos\left(\frac{C-D}{2}\right)\\
\indent\indent \smallskip sinC-sinD=2\,cos\left(\frac{C+D}{2}\right)sin\left(\frac{C-D}{2}\right)\\
\indent\indent \smallskip cosC+cosD=2\,cos\left(\frac{C+D}{2}\right)cos\left(\frac{C-D}{2}\right)\\
\indent\indent \smallskip cosC-cosD=2\,sin\left(\frac{C+D}{2}\right)cos\left(\frac{D-C}{2}\right)
$
\subsubsection{Special Cases:}
$
\displaystyle
\indent\indent \,\,\, \smallskip tan\left(\frac{\pi}{4}+\theta\right)=\frac{1+tan\theta}{1-tan\theta}\\
\indent\indent \,\,\, tan\left(\frac{\pi}{4}-\theta\right)=\frac{1-tan\theta}{1+tan\theta}
$
\subsection{Product Formulae:}
$
\indent\indent 2\,sinA\,cosB=sin(A+B)+sin(A-B)\\
\indent\indent 2\,cosA\,sinB=sin(A+B)-sin(A-B)\\
\indent\indent 2\,cosA\,cosB=cos(A+B)+cos(A-B)\\
\indent\indent \smallskip 2\,sinA\,sinB=cos(A-B)-cos(A+B)\\
\indent\indent sin^{2}A-sin^{2}B=sin(A+B)\,sin(A-B)\\
\indent\indent cos^{2}A-sin^{2}B=cos(A+B)\,cos(A-B)
$
\subsection{Double Angle Formulae:}
$
\indent\indent \smallskip sin\,2\theta=2\,sin\theta \, cos\theta\\
\displaystyle
\indent\indent \medskip sin\,2\theta=\frac{2\,tan\theta}{1+tan^{2}\theta}\\
\indent\indent cos\,2\theta=cos^{2}\theta-sin^{2}\theta\\
\indent\indent cos\,2\theta=2\,cos^{2}\theta-1\\
\indent\indent cos\,2\theta=1-2\,sin^{2}\theta\\
\indent\indent \medskip cos\,2\theta=\frac{1-tan^{2}\theta}{1+tan^{2}\theta}\\
\indent\indent tan\,2\theta=\frac{2\,tan\theta}{1-tan^{2}\theta}\\
$
\subsection{Triple Angle Formulae:}
$
\indent\indent sin\,3\theta=3\,sin\theta-4\,sin^{3}\theta\\
\indent\indent cos\,3\theta=4\,cos^{3}\theta-3\,cos\theta\\
\displaystyle
\indent\indent tan\,3\theta=\frac{3\,tan\theta-tan^{3}\theta}{1-3\,tan^{2}\theta}
$
\subsection{Miscellaneous:}
$
\indent\indent sin(-\theta)=-sin\theta\\
\indent\indent cos(-\theta)=cos\theta\\
\indent\indent tan(-\theta)=-tan\theta
$
\subsubsection{Identities:}
$
\indent\indent \,\,\, sin^{2}\theta+cos^{2}=1\\
\indent\indent \,\,\, sec^{2}\theta-tan^{2}\theta=1\\
\indent\indent \,\,\, tan^{2}\theta-cot^{2}\theta=1
$
\end{document} 
